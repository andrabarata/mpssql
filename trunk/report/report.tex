\documentclass[12pt]{article}

\usepackage[russian]{babel}
\usepackage[utf8]{inputenc}
\usepackage{color}
\usepackage{listings}
\definecolor{Brown}{cmyk}{0,0.81,1,0.60}
\definecolor{OliveGreen}{cmyk}{0.64,0,0.95,0.40}
\definecolor{CadetBlue}{cmyk}{0.62,0.57,0.23,0}
\lstset{language=Java,
 frame=ltrb,framesep=2pt,
 basicstyle=\normalsize,
 keywordstyle=\ttfamily\color{CadetBlue},
 commentstyle=\color{Brown},
 stringstyle=\ttfamily\color{OliveGreen},
 breaklines=true
 breakatwhitespace=true
 showstringspaces=ture}

\title{Курсовая работа\\
Руководитель: Вадим Гуров\\
(кандидат наук, программист ООО "ИнтеллиДжей Лабс")\\
Интеграция языка для работы с реляционными базами данных в базовый язык JetBrains MPS}
\author{Выполнил Никитин Павел Антонович\\
Кафедра системного программирования СПбГУ}
\date{9 мая 2009}

\begin{document}
\maketitle	

\section{Введение}
Для разработчиков программного обеспечения из разных доменов могут быть полезными всевозможные доменно-специфичные расширения языков программирования общего назначения. Например, разработчики приложений для банковских нужд оценят встроенную в язык поддержку работы с денежными единицами. К сожалению, традиционные текстовые языки обязаны обладать однозначной грамматикой, что делает их трудно расширяемыми. Целью данной работы было именно расширение языка общего назначения (Java) и добавление в него конструкций для работы с реляционными базами данных. Для решения проблемы возможной неоднозначности грамматики полученного языка был выбран языковой инструментарий JetBrains MPS (Meta Programming System) -- средство для создания языков и интегрированная среда для разработки на них. MPS решает эту проблему, работая непосредственно с абстрактным синтаксическим деревом программы, для редактирования которого используется текстово-подобный проекционный редактор. Однако, языково-ориентированное программирование в таком виде, как оно существует сегодня -- достаточно молодая парадигма. Например, как другой заметный его представитель -- Intentional Software, так и MPS выходят из beta только в этом году.\\

\section{О работе с реляционными БД}
Тем не менее, потребность удобной работы с базами данных возникла давно. Поэтому существует масса ad-hoc решений для различных языков программирования. Самый известный из них -- так называемый Embedded SQL. При таком подходе программа должна быть обработана специальным препроцессором, прежде чем она будет откомпилирована компилятором базового языка программирования. Препроцессор распознаёт вызовы запросов внутри языковых предложений и преобразует их в библиотечные вызовы, также распознаются специальным образом оформленные ссылки на переменные базового языка внутри SQL-предложений и некоторые другие менее фундаментальные конструкции -- т. е. происходит трансляция. Данный подход доступен для языков C, Ada, Cobol, Fortran.\\

Похожий подход предлагает и стандарт SQLJ для языка Java. В этом стандарте в качестве конечной библиотеки так или иначе используется JDBC, входящая в стандартную поставку от Sun начиная с JDK версии 1.1, но чаще всего она обёрнута в пропиетарный код, как, например, в реализации SQLJ от Oracle. Пример кода на SQLJ:
\begin{lstlisting}
int i, j;
i = 1;
#sql {	SELECT field INTO :OUT j
          FROM table
          WHERE id = :IN i };
System.out.println(j);
\end{lstlisting}
О JDBC мы поговорим подробнее, так как именно она является основой для многих решений, работающих с базами, например Hibernate. В принципе, для несложных приложений вполне применимо использование JDBC напрямую, без каких-либо обёрток, поэтому наряду с Embedded SQL будем рассматривать и решения, не использующие препроцессор.\\

Все эти решения объединяет одно -- слабо выраженная структура. Тем не менее, они реально используются для разработки программного обеспечения, и в процессе общения с ними разработчик может столкнуться с вытекающими отсюда трудностями. Во-первых, это нарушение статического синтаксиса запроса, например -- опечатка в ключевом слове SELECT. Класс решений с препроцессором позволет найти такую ошибку на стадии трансляции, что в принципе, приемлимо. Однако строки, явно попадающие в параметры функций JDBC не обрабатываются препроцессором и, следовательно, ошибка будет замечена только во время выполнения и приведёт к исключению SQLException, что неприемлимо даже при наличии покрытия тестами части кода, содержащей ошибку, так как отвлекает разработчика от решения актуальных задач. Во-вторых, это проблема вложенности запросов. В JDBC конструкция с вложением языков вида $Java_1[SQL_1[Java_2[SQL_2]]]$ может привести к трудно воспроизводимому нарушению синтаксиса формируемого запроса $SQL_1 \cup SQL_2$, если на уровне вложенности $Java_2$ используется нетривиальное ветвление. SQLJ же вовсе исключает возможность такой вложенности.\\

Ещё один интересный подход - создание, в отличие от SQLJ, не внешнего доменно-специфичного языка (со своим расширенным синтаксисом и другой грамматикой), а внутреннего, то есть использование средств базового языка (Java) для придания коду на нём подобия специального языка (для работы с реляционными БД в нашем случае). Данный подход реализуют такие библиотеки, как, например, jequel, squill, quaere. Пример кода на jequel:
\begin{lstlisting}
final ARTICLE ARTICLE2 = ARTICLE.as("article2");
final Field OID = ARTICLE_C.OID.as("article_c_oid");

SqlString sql = select(ARTICLE2.OID, ARTICLE_C_OID)
                 .from(ARTICLE2, ARTICLE_C)
                 .where(ARTICLE2.OID.eq(OID));
\end{lstlisting}

Особняком стоит недоступное для Java-мира расширение языка C\# LINQ to SQL, но его использование предполагает знание своего SQL-подобного синтаксиса помимо знания LINQ и SQL.\\

\section{Значимость IDE}
Конечно, наличия ошибок в коде можно избежать при достаточной дисциплине со стороны разработчика. Но для существенного повышения производительности используются интегрированные среды разработки с структурированной подсказкой вводимого кода, подсветкой ошибок, анализом потока управления, рефакторингом и прочими полезными функциями. Однако популярные среды разработки на Java, такие, как Eclipse и IntelliJ IDEA не поддерживают Embedded SQL ни самостоятельно, ни в качестве плагинов. Есть некая среда JDeveloper от Oracle, но её SQLJ привязан к их базе данных и их пропиетарной Java-библиотеке. Но, даже если бы плагины для SQLJ для популярных сред были разработаны

\section{Основная часть}
Вообще абзац.

\section{Список литературы}
Ёж.

\end{document}